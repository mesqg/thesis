\chapter{Problem statement and planning}
\label{planning}
\section{Problem statement}
What I wrote in November remains valid:
\begin{itemize}
\item ``This  thesis  is  a  study  of  implicit  type conversions.  The intention is to understand how this mechanism is already employed in Scala, adapt it to have arguably better properties and implement a (prototype) version of it in the KU Leuven's Haskell compiler.  The goal is to improve on Scala's approach by supporting transitivity in the implicit conversions.   Important  aspects  of  this  thesis  are:  how  to  ensure  that  the resolution process remains unambiguous in the presence of transitivity and how to make this extension as easy as possible to work with.''
\end{itemize}
We want to allow for implicit conversions with parametric polymorphism and constraints, much in the same way of Haskell's type classes. We are working on a short but powerful set of rules that can express that ``$\forall \tyVar_1 . \forall \tyVar_2 . (\tyVar_1 \rightsquigarrow \tyVar_2 => T \; \tyVar_1 \rightsquigarrow T \;\tyVar_2)$'', for some specific type constructor $T$.

For some cases such as List, we'd like to have these implicit conversion schemes pre-loaded in the compiler.
\section{Planning}
The first priority in my thesis now is to finish the syntax directed typing rules. This were written for a language in which implicit conversions are globally scoped and that doesn't allow any ambiguity to be present in the implicit environment.

I then plan to study the KUL's Haskell compiler and implement this extension on it. Time permitting, I would like to study other alternatives: a language with local scoping and/or one which allows for several conversion paths within the same scope but assigns costs to each and chooses the one with lowest.

The algorithms designed should be rigorously studied and the object of proofs about their correctness. I also plan to continuing my study of the relevant literature.

This effort to write the intermediate text made me realize that the writing of the thesis text must continue from the start of the second semester.

The gantt (\fref{gantt}) chart shows this planning.
\begin{figure}
  \centering
  \includegraphics[height=8cm]{gantt}
  \label{gantt}
  \caption{My planned time line for the second semester}
\end{figure}


%%% Local Variables: 
%%% mode: latex
%%% TeX-master: "thesis"
%%% End: 
